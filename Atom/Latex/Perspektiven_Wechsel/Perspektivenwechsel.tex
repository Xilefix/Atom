\documentclass[12pt]{scrartcl}

% Kodierung dieser Datei angeben
\usepackage[utf8]{inputenc}

% Schönere Schriftart laden
\usepackage[T1]{fontenc}
\usepackage{lmodern}

% Deutsche Silbentrennung verwenden
\usepackage[ngerman]{babel}

% Anführungszeichen mit \enquote
\usepackage{csquotes}

% Bessere Unterstützung für PDF-Features
\usepackage[breaklinks=true]{hyperref}

\KOMAoptions{%
  % Absätze durch Abstände
  parskip=full,%
  % Satzspiegel berechnen lassen
  DIV=calc%
}

\begin{document}
Situation in 11. Klasse Info:

Der Unterricht wird in einer Klasse mit 15 Schülern gehalten. Die Stunde findet nachmittags um 14 Uhr statt. Der Lehrer hat die Klasse in 2 Gruppen eingeteilt. Zwei Schüler lösen in Stillarbeit Aufgaben eines schwierigeren Anforderungsniveaus. Die übrigen Schüler erhalten die normalen Aufgaben des Lehrers. Jeder Schüler hat eine unterschiedliche Aufgabe, die er in Selbstarbeit lösen soll.

Während die Aufgaben bearbeitet werden, geht der Lehrer in der Klasse herum und begutachtet den Fortschritt der Schüler. Er fordert einen Schüler auf, ein Verfahren mit Worten zu beschreiben, wie bei einem eingegebenen Passwort dessen Asciicodesumme berechnet wird.
Der Schüler schreibt seine Lösung auf ein Blatt Papier. Er formuliert:  \enquote{Es wird ein Passwort eingegeben, dessen Typ Integer ist}.
Der Lehrer sieht die Schülerlösung bei seinem Rundgang.
Der Lehrer spricht lauter als gewöhnlich zu der ganzen Klasse (bis auf erste Gruppe): \enquote{Wie sollt ihr die Prüfung schreiben, wenn ihr nicht einmal wisst, welchen Typ ein Passwort hat?} Die anderen Schüler sehen zum Lehrer hin. Es erfolgt keine Meldung. Daraufhin sagt der Lehrer: \enquote{Ich spreche nicht nur mit dir} (meint den Schüler)
\enquote{Welchen Typ hat ein Passwort? Kann ein Passwort eine Ganze Zahl sein?} Schüler: \enquote{Dann Character?} Lehrer: \enquote{Habt ihr euch die Unterlagen zum Delphi-Programmieren überhaupt angeschaut? Welchen Variablentyp hat eine Zeichenkette?} Der Schüler hat Tränen in den Augen und antwortet nicht.
Der betroffene Schüler, der direkt neben mir sitzt, fragt mich leise: \enquote{String oder? Aber es kann doch auch nur ein Pin sein.} (Die Antwort des Schülers ist richtig, wird aber vom Lehrer und der Klasse nicht gehört.) Der Lehrer nennt der Klasse als Lösung String.

Später erzählt mir der Lehrer, dass er erst später seinen Fehler bemerkt habe, weil er nicht an Handypasswörter (Pin) gedacht habe.


Perspektivenwechsel (Denken, Fühlen, Wollen):

Akteure: der Schüler, Mitschüler, Lehrer

Der Schüler:
\begin{itemize}
  \item denkt:
  \begin{itemize}
    \item Was will der Lehrer von mir eigentlich?
    \item Bin ich dumm?
    \item Was habe ich falsch gemacht?
    \item Wer kann mir helfen?
    \item Werde ich vorgeführt?
    \item Bin ich mit dem Problem alleine?
    \item Wieso ist der Lehrer so?
    \item Was denken die anderen Schüler von mir?
    \item Ich habe doch mein bestes gegeben
    \item Werde ich Informatik bzw. Variablen irgendwann mal verstehen?
    \item Wieso versteht der Lehrer mich nicht?
    \item Was habe ich dem Lehrer getan?
    \item Der lehrer ist immer so!
    \item Was ist an meiner Antwort falsch?
    \item Handy-Pin ist doch Integer?!
  \end{itemize}
  \item fühlt:
  \begin{itemize}
    \item bedrückt
    \item missverstanden
    \item vorgeführt
    \item traurig
    \item verärgert
    \item hilflos
    \item überwältigt
    \item traumatisiert
  \end{itemize}
  \item will:
  \begin{itemize}
  \item vom Lehrer verstanden werden
  \item von den anderen Mitschülern verstanden werden
  \item von den anderen Mitschülern nicht als dumm betrachtet werden
  \item das Problem verstehen
  \item aus der Situation raus
  \item nicht weiter auffallen
  \item dass der Unterricht zuende ist
  \item nicht im Mittelpunkt stehen
  \end{itemize}
\end{itemize}
Die Mitschüler:
\begin{itemize}
  \item denken:
  \begin{itemize}
    \item Wegen dem ist der Lehrer auf uns alle sauer!
    \item Kapiert der nie was?!
    \item War seine Antwort nicht richtig, jetzt verstehe ich nichts mehr?!
    \item Übertreibt der Lehrer nicht?
    \item Soll ich dem Lehrer wirklich zuhören
    \item Der Schüler weint doch schon, hör doch endlich auf
  \end{itemize}
  \item fühlen:
  \begin{itemize}
    \item empfinden Belustigung
    \item empfinden Mitleid
    \item sind genervt
    \item Angst, selber drann zu kommen
    \item vom Lehrer beleidigt/angegriffen
  \end{itemize}
  \item wollen:
  \begin{itemize}
    \item nicht vom Lehrer selber angesprochen werden
    \item unsichtbar werden
    \item eine plausible Lösung des Lehrers
    \item weiter arbeiten
  \end{itemize}
\end{itemize}
Lehrer:
\begin{itemize}
  \item denkt:
  \begin{itemize}
    \item Bei diesem Schüler ist meine Mühe umsonst!
    \item Der schafft es sowieso nicht!
    \item Die ganze Klasse bringt micht zur Verzweiflung
    \item Bald geh ich eh in Pension
    \item Ist denn hier überhaupt einer, der es versteht?
    \item Wie kann ich das Kursergebnis noch retten?
    \item Wie kann ich die Schüler motivieren?
  \end{itemize}
  \item fühlt:
  \begin{itemize}
    \item unfähig
    \item überfordert
    \item im Recht
    \item zornig
    \item wütend
    \item verzweifelt
    \item hilflos
    \item überlegen
  \end{itemize}
  \item will:
  \begin{itemize}
    \item motiviertere Schüler
    \item lernende Schüler
    \item fleißigere Schüler
    \item schlauere Schüler
    \item Lösung im Plenum klären
    \item richtige Vermittlungsstrategie
    \item Unterschied der Variablentypen erklären
  \end{itemize}
\end{itemize}
\end{document}
