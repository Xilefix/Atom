\documentclass[12pt]{scrartcl}

% Kodierung dieser Datei angeben
\usepackage[utf8]{inputenc}

% Schönere Schriftart laden
\usepackage[T1]{fontenc}
\usepackage{lmodern}

% Deutsche Silbentrennung verwenden
\usepackage[ngerman]{babel}

% Bessere Unterstützung für PDF-Features
\usepackage[breaklinks=true]{hyperref}

\KOMAoptions{%
  % Absätze durch Abstände
  parskip=full,%
  % Satzspiegel berechnen lassen
  DIV=calc%
}

% TikZ laden
\usepackage{tikz}

% Verwendete TikZ-Bibliotheken laden
\usetikzlibrary{positioning,automata}
\title{Informatik ist mehr als Programmieren}
\author{Felix Liesendahl}
\begin{document}
\maketitle
In der Bevölkerung und auch bei Schülern, die am Informatikunterricht und/oder Studium interessiert sind, besteht die weitverbreitete Meinung, das Schwerpunkt der Informatik das Programmieren sei.
Gerade Schüler, die vor der Wahl des Faches Informatik stehen, erhoffen sich Kenntnisse zum Programmieren von Spielen oder dem Erstellen von Internetauftritten und übersehen dabei viele Anwendungsbereiche der Informatik.

Ebenso wie das Fach Mathematik hat aber auch das Fach Informatik eine für die Allgemeinbildung (einschließlich der politischen Bildung) unentbehrliche Funktion und müsste daher nicht nur als Wahlfach, sondern für alle Schüler verpflichtend unterrichtet werden.

Der moderne Informatikunterricht muss Grunderfahrungen ermöglichen, die vielfältig miteinander verknüpft sind, ohne die ein Verständnis der heutigen digitalen Welt, in der wir alle leben, nicht möglich ist. Drei Grunderfahrungen werden als Unterrichtsergebnis erwartet:

\begin{enumerate}
 \item Zuerst müssen Informatiksysteme in allen Lebensbereichen bewusst wahrgenommen werden. Diese müssen dann verstanden und bewertet werden.
 \item Es muss erfahren werden, dass sich menschliche Handlungen, die zu einer Problemlösung führen, auf einen Computer übertragen lassen. Dazu gehört auch die Erkenntnis, dass die Realität in einem Informatiksystem abgebildet werden kann und dass es auch unterschiedliche Problemlösungen geben kann.
 \item Wenn im Unterricht Kompetenzen entwickelt werden, muss immer gleichzeitig die Vorstellung damit verknüpft sein, wo diese Kompetenzen angewandt werden können.
 Nur wenn dies gegeben ist, der Schüler somit weiß, dass sich seine Kompetenzen auch in andere Fächern und außerhalb der Schule anwenden lassen, wird er motiviert dem Unterricht folgen und ein realistisches Tätigkeitsbild erlangen.
\end{enumerate}

Bei einer traditionellen Unterrichtsstunde zum Lehrplanthema \textit{Arbeitsweise von Such- und Sortieralgorithmen erläutern} wird das Sortieren als Alltagsproblem aufgezeigt, indem z. B. die Schüler Karten oder sich selbst sortieren müssen, was mühelos gelingt. Danach werden die Beschränkungen des Computers vermittelt. Jetzt erkennen die Schüler erst ein Problem. Zur Ergebnissicherung werden schließlich vom Lehrer noch eine oder mehrere algorithmische Lösungen gezeigt (Quicksort oder Bubblesort).

Beim neuen Ansatz \glqq Sortieren -- Wo ist das Problem?\grqq{} ist zum Beispiel beim Sortieren von Filmrollen, die für die Schüler nicht sichtbar mit unterschiedlichen Mengen an Kupfermünzen gefüllt sind und deren Gewicht durch eine Waage ermittelt wird, das Problem nicht äußerlich sichtbar.
Das Sortier\underline{merkmal} wird erst durch den Handlungsvorgang (Wiegen) erkennbar.
Es ergibt sich der didaktische Vorteil, dass das Sortieren direkt als Computerproblem erkannt wird.
Durch die Handlungsorientierung wird das Verständnis gefördert. Das Problembewusstsein der Schüler soll so intensiver und zu einem früheren Zeitpunkt als beim traditionellen Ansatz erzeugt werden.
Anschließend werden auch bei diesem Ansatz die Sortieralgorithmen als Verständniskontrolle erläutert.

\end{document}
