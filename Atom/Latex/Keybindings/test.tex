\documentclass{scrartcl}

% Kodierung dieser Datei angeben
\usepackage[utf8]{inputenc}

% Schönere Schriftart laden
\usepackage[T1]{fontenc}
\usepackage{lmodern}

% Deutsche Silbentrennung verwenden
\usepackage[ngerman]{babel}

% Bessere Unterstützung für PDF-Features
\usepackage[breaklinks=true]{hyperref}

\KOMAoptions{%
  % Absätze durch Abstände
  parskip=full,%
  % Satzspiegel berechnen lassen
  DIV=calc%
}

% TikZ laden
\usepackage{tikz}

% Verwendete TikZ-Bibliotheken laden
\usetikzlibrary{positioning,automata}

\begin{document}
\tableofcontents
  \begin{tikzpicture}[auto, thick]
    \node[initial, state] (q0) {$q_0$};
    \node[state, right=of q0] (q1) {$q_1$};
    \node[state, accepting, right=of q1]
      (q2) {$q_2$};
    \path (q0) edge[->] node {0} (q1)
          (q1) edge[->, loop above] node {0} ()
               edge[->, bend left] node {1} (q2)
          (q2) edge[->, bend left] node {0} (q1);
  \end{tikzpicture}

  \section{Kapitel}
  test zeilenumbruch\\
  testende

  \textbf{test zeilenumbruch}

testende

  \subsection{Subsection}
  \begin{enumerate}
    \item
    ctrl+shift +7 für Auskommentieren an/aus
    \item
    \(\$\$\) für displaymath und nur eins für inlinemath
    \item
    ctrl+alt+7 für \(\{\}\) um Markierung
    \item
    test

  \end{enumerate}

\begin{description}
  \item[Gummibär] gummi
  \item[ich] person
\end{description}
  \subsubsection{Subsub}
  % h als option steht für die Positon wo es definiert wurde
  \begin{table}[h]
    \centering
    \begin{tabular}{l|ll}
      \textbf{Jahr} & \textbf{D} & \textbf{F}
      \\  \hline
      1990 & 34 & 54  \\
      2000 & 64 & 89
    \end{tabular}
    \caption{Jahreszahlen}
    \label{tbl:einwohner}
  \end{table}

\end{document}
