\documentclass[12pt]{scrartcl}

% Kodierung dieser Datei angeben
\usepackage[utf8]{inputenc}

% Schönere Schriftart laden
\usepackage[T1]{fontenc}
\usepackage{lmodern}

% Deutsche Silbentrennung verwenden
\usepackage[ngerman]{babel}

% Anführungszeichen mit \enquote
\usepackage{csquotes}

% Bessere Unterstützung für PDF-Features
\usepackage[breaklinks=true]{hyperref}

\KOMAoptions{%
  % Absätze durch Abstände
  parskip=full,%
  % Satzspiegel berechnen lassen
  DIV=calc%
}

\begin{document}
Situation in 11. Klasse Info:

Der Unterricht wird in einer Klasse mit 15 Schülern gehalten. Die Stunde findet nachmittags um 14 Uhr statt. Der Lehrer hat die Klasse in 2 Gruppen eingeteilt. Zwei Schüler lösen in Stillarbeit Aufgaben eines schwierigeren Anforderungsniveaus. Die übrigen Schüler erhalten die normalen Aufgaben des Lehrers. Jeder Schüler hat eine unterschiedliche Aufgabe, die er in Selbstarbeit lösen soll.

Während die Aufgaben bearbeitet werden, geht der Lehrer in der Klasse herum und begutachtet den Fortschritt der Schüler. Er fordert einen Schüler auf, ein Verfahren mit Worten zu beschreiben, wie bei einem eingegebenen Passwort dessen Asciicodesumme berechnet wird.
Der Schüler schreibt seine Lösung auf ein Blatt Papier. Er formuliert:  \enquote{Es wird ein Passwort eingegeben, dessen Typ Integer ist}.
Der Lehrer sieht die Schülerlösung bei seinem Rundgang.
Der Lehrer spricht lauter als gewöhnlich zu der ganzen Klasse (bis auf erste Gruppe): \enquote{Wie sollt ihr die Prüfung schreiben, wenn ihr nicht einmal wisst, welchen Typ ein Passwort hat?} Die anderen Schüler sehen zum Lehrer hin. Es erfolgt keine Meldung. Daraufhin sagt der Lehrer: \enquote{Ich spreche nicht nur mit dir} (meint den Schüler)
\enquote{Welchen Typ hat ein Passwort? Kann ein Passwort eine Ganze Zahl sein?} Schüler: \enquote{Dann Character?} Lehrer: \enquote{Habt ihr euch die Unterlagen zum Delphi-Programmieren überhaupt angeschaut? Welchen Variablentyp hat eine Zeichenkette?} Der Schüler hat Tränen in den Augen und antwortet nicht.
Der betroffene Schüler, der direkt neben mir sitzt, fragt mich leise: \enquote{String oder? Aber es kann doch auch nur ein Pin sein.} (Die Antwort des Schülers ist richtig, wird aber vom Lehrer und der Klasse nicht gehört.) Der Lehrer nennt der Klasse als Lösung String.

Später erzählt mir der Lehrer, dass er erst später seinen Fehler bemerkt habe, weil er nicht an Handypasswörter (Pin) gedacht habe.
\end{document}
