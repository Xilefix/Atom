\documentclass[11pt]{scrartcl}

% Kodierung dieser Datei angeben
\usepackage[utf8]{inputenc}

% Schönere Schriftart laden
\usepackage[T1]{fontenc}
\usepackage{lmodern}
\usepackage {units}
\usepackage{csquotes}

% Tabelle im Querformat
\usepackage{lscape}
\usepackage{pdflscape}


% Deutsche Silbentrennung verwenden
\usepackage[ngerman]{babel}

% Bessere Unterstützung für PDF-Features
\usepackage[breaklinks=true]{hyperref}

\KOMAoptions{%
  % Absätze durch Abstände
  parskip=full,%
  % Satzspiegel berechnen lassen
  DIV=calc%
}

% TikZ laden
\usepackage{tikz}

% Verwendete TikZ-Bibliotheken laden
\usetikzlibrary{positioning,automata}
\title{Klasse 10b Einführung in Verknüpfungen von Ereignissen und Vierfeldertafel}
\author{Felix Liesendahl}
\begin{document}
\maketitle
\tableofcontents
\newpage

% 1. Allgemeine Fakten zur Lerngruppe
% - Schule, Schulart
% - Klasse, Kurs
% - Anzahl der SuS
% - soziale Zusammensetzung
% - Sozialverhalten
% - Lernklima, Lernmotivation
% - Lernverhalten
% - verhaltensproblematische SuS
% - SuS mit Nachteilsausgleich
% - SuS mit Gutachten
% - Methodenkompetenz
% - ……….
% 2. Äußere Bedingungen
% - Ausstattung des Raumes
% - digitale Medien
% - Sitzordnung
% - Lernbegleiter
% - Verortung der Stunde im Tagesablauf
% - ………….
% 3. Fachliche Lernvoraussetzungen
% - allgemeiner Leistungsstand in Mathematik
% - vorangegangener Unterricht und folgende Unterrichtsinhalte
% - Vorwissen
% - Akzeptanz verschiedener Unterrichtsformen und – Methoden
\section{Lernvoraussetzungen}
\begin{enumerate}
  \item  Allgemeine Fakten zur Lerngruppe:
  Der Unterricht soll in der Klasse 10b des Adolf-Reichwein-Gymnasiums stattfinden.
  Die Klasse wurde wegen Corona in zwei Gruppen je 10 Schüler unterteilt.
  Es gibt keine verhaltensproblematische oder ähnliche SuS.
%   \begin{itemize}
%     \item Schulart: Gymnasium
%     \item Klasse, Kurs: 10, b
%     \item Anzahl der SuS: 2 Gruppen je 10 Schüler
%     \item verhaltensproblematische SuS: keine
%       \end{itemize}

  \item Äußere Bedingungen:
  Der Raum besitzt eine Tafel, Beamer mit einem angeschlossenen Laptop. Die Verortung der Stunde im Tagesablauf wird vermutlich in der 3. oder 4. Stunde sein.\\
%   \begin{itemize}
%     \item Ausstattung des Raumes: Tafel, Beamer, Laptop
%     \item digitale Medien: Beamer
Sitzordnung:
    \begin{center}
\begin{tabular}{lll}
 x- & x- & -x\\
x - & -x & - -\\
-x & x- & x-\\
-x & -x & - -\\
- - & - - & - -\\
    &    & X\\
    &  T  & B
    \end{tabular}
    \end{center}
    Erläuterung der Sitzordnung: großes X steht für das Lehrerpult, kleine x sind die Schüler sowie T für Tafel und B für Beamer
%     \item Verortung der Stunde im Tagesablauf: noch unbekannt vermutlich in der 3. oder 4. Stunde
%   \end{itemize}
  \item Fachliche Lernvoraussetzungen:
  Im vorangegangenen Unterricht haben die SuS bereits die Themen Ereignisse, Wahrscheinlichkeiten von Ereignissen, Urnenmodelle, Erwartungswerte mit deren diskreten Wahrscheinlichkeitsverteilungen (erste Grundlagen ohne Spezialfälle) und Standardabweichungen.\\
  Von den SuS kann daher erwartet werden, dass sie Ereignisse kennen und deren Wahrscheinlichkeiten berechnen können. Die SuS können mit Baumdiagrammen umgehen, kennen aber noch nicht die Vierfeldertafel.
  Nach der geplanten Unterrichtsstunde soll das Thema diskrete Wahrscheinlichkeitsverteilungen vertieft durchgenommen werden.
%   \begin{itemize}
%     \item  allgemeiner Leistungsstand in Mathematik:
%     \item vorangegangener Unterricht: Ereignisse, Wahrscheinlichkeiten, Urnenmodelle, Erwartungswerte, Standardabweichungen
%     \item folgende Unterrichtsinhalte: Wahrscheinlichkeitsverteilungen
%     \item Vorwissen: SuS kennen Ereignisse, können Wahrscheinlichkeiten von Ereignissen berechnen
%   \end{itemize}
\end{enumerate}

\section{Lernziele}
\begin{enumerate}
  \item SuS können Ergeignisse verknüpfen und durch Symbole (z.B.: \(A \cup B,A \cap B, \bar A\)) beschreiben (Sachkompetenz).
  \item SuS können zweistufige Zufallsexperimente mit Hilfe einer Vierfeldertafel veranschaulichen und die Absoluten Werte der Felder bestimmen (Sachkompetenz).
  \item SuS lernen mathematisch zu modellieren (K3) (Methodenkompetenz).
  \item SuS können die Ergebnisse stochastischer Berechnungen auf Plausibilität prüfen (Methodenkompetenz).
\end{enumerate}

% Überschriften mit Absätzen
\newpage
\begin{landscape}
\section{Struktur der Stunde (Phasen)}
\textbf{Klasse: 10b} \textbf{Datum: Juni} \textbf{Lehrer: Herr Liesendahl bei Frau Danckwerts}\\
\textbf{Stundenthema:Stochastik: Einführung der Verknüpfung von Ereignissen und Vierfeldertafel}\\
\textbf{Lernziele: Ergeignisse verknüpfen, Symbole (z.B.: \(A \cup B,A \cap B, \bar A\)) beschreiben, zweistufige Zufallsexperimente mit Hilfe einer Vierfeldertafel veranschaulichen, die Absoluten Werte der Felder bestimmen, mathematisch modellieren (K3),  Ergebnisse stochastischer Berechnungen auf Plausibilität prüfen}
\begin{center}
\begin{tabular}{|c|c|c|c|c|c|}\hline
Zeit min.& Phase & Lehrertätigkeit + mathem. Inhalt & Schülertätigkeit & Methode/Sozialform & Medien\\\hline
2& Einf. & L begrüßt Klasse und führt in das Thema ein: & S hören zu & Lehrervortrag & Beamer\\\hline
10 &  & Wdh. Ereignisse verknüpfen (siehe Powerpoint)  &  & Lehrer-Schüler-Gespräch & \\\hline
2& Hauptt.& AB: Kopie der Seite 135 den Schülern austeilen  &  &  & AB\\\hline
 10&  &  Aufgabe an SuS: In Einzelarbeit durchzulesen  & S machen Aufgabe & selbstständige Arbeit & \\\hline
 &  & erklären zu können &  &  & \\\hline
 &  & Parallel dazu L erstellt Tafelbild&  &  & Tafel\\\hline
5 &  & Besprechung Tafelbild & S stellen Fragen & im Plenum & Tafel\\\hline
3 &  &  & S schr. Tafel ab &  & Tafel\\\hline
12 &  & Bsp medizinischer Test (S. 135) im Plenum &  lesen, antworten&im Plenum  & Buch\\\hline
 1&  & Erteilung der HA: S.136 Nr. 1 und 2 & schreiben HA auf &  & \\\hline
\end{tabular}
\end{center}
\end {landscape}



\section{Arbeitsanweisungen/Gelenkstellen}


\section{Tafelbild}


\section{Methodenwahl}
Diverse Wechsel der Methoden zu Lockerung des Unterrichtsgeschehens. Alternative wäre gewesen, erst ein Beispiel mit Zahlen vorzustellen und daran die neue Methode zu erläutern und die theoretischen Grundlagen daraus abzuleiten.
Die Unterrichtsziele sind erfüllt, wenn die SuS ein einfaches Beispiel, wie die HA Nr. 1 selbstständig lösen und verstehen können.


\section{Arbeitsmaterial}
Buch Lambacher Schweitzer Seite 135/136\\
Powerpointpräsenmtation\\
Tafelbild\\
siehe Anhang


\section{Reflexion und Schlussfolgerung für die eigene Arbeit}
Das Thema war viel anspruchsvoller als erwartet, da die Unterrichtsstunde zu wenig Zeit zum Unterrichten bietet, um ausführlich Grundlagen zu schaffen.










\end{document}
